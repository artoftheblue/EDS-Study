\documentclass[a4paper, 10pt]{article}
\usepackage{header}

\title{\LARGE{Математический анализ—2}}
\author{Винер Даниил, Хоранян Нарек}
\date{Версия от \today}

\begin{document}
\maketitle
\tableofcontents
\newpage
\setlength{\parindent}{15pt}
\setlength{\parskip}{2mm}
% \section{\textbf{Формула оценки}}
% $$O_{\text{Итог}} = \min(round(0.15*\text{ДЗ} + 0.2*\text{Коллоквиум} + 0.22*\text{КР} + 0.35*\text{Э} + 0.03*\text{Листочки} + 0.1*\text{Лаба}), 10)$$
\section{Кратные интегралы. Брусья. Интегрируемые функции по Риману}
\subsection{Брус. Мера бруса}
\definition Замкнутый брус (координатный промежуток) в $\mathbb{R}^n$ — множество, описываемое как
\begin{equation*}
\begin{aligned}
    I&=\{x\in\mathbb{R}^n\ |\ a_i\leqslant x_i\leqslant q_i,\ i\in\{1,n\}\}\\
    &=\left[a_1,b_1\right]\times\ldots\times\left[a_n,b_n\right]
\end{aligned}
\end{equation*}
\comment $I=\{a_1,b_1\}\times\ldots\times\{a_n,b_n\}$, где $\{\}$ может быть отрезком, интервалом и т.д.

\definition Мера бруса — его объём:
\begin{equation*}
    \begin{aligned}
        \mu(I)&=|I|
        =\prod_{i=1}^{n} (b_i-a_i)
    \end{aligned}
\end{equation*}

\subsection{Свойства меры бруса в $\R^n$}
\begin{enumerate}
    \item \textbf{Однородность:} $\mu(I_{\lambda a,\lambda b})=\lambda^n\cdot\mu(I_{a,b})$, где $\lambda\geqslant
    0$
    \item \textbf{Аддитивность:} Пусть $I, I_1, \ldots, I_k$ — брусы
    
    Тогда, если $\forall i, j\, I_i, I_j$ не имею общих внтренних точек, и $\displaystyle\bigcup_{i=1}^kI_i = I$, то
    $$|I| = \sum_{i=1}^k|I_i|$$
    \item \textbf{Монотонность}: Пусть $I$ — брус, покрытый конечной системой брусов, то есть $I\subset \displaystyle\bigcup_{i=1}^kI_i$, тогда
    $$|I| < \sum_{i=1}^k|I_i|$$
\end{enumerate}
\subsection{Разбиение бруса. Диаметр множества. Масштаб разбиения}
\definition \label{1.3} $I$ — замкнутый, невырожденный брус и $\displaystyle\bigcup_{i=1}^kI_i = I$, где $I_i$ попарно не имеют общих внутренних точек. Тогда набор $\T = \{\T\}_{i=1}^k$ называется разбиением бруса $I$

\definition \label{1.4} Диаметр произвольного ограниченного множества $M\subset\R^n$ будем называть 
\begin{equation*}
\begin{aligned}
    d(M) = \displaystyle\sup_{1\leqslant i\leqslant k}\|x-y\|,\text{ где}\\
    \|x-y\|=\sqrt{\sum_{i=1}^{n}\left(x_i-y_i\right)^2}
\end{aligned}
\end{equation*}

\definition \label{1.5} Масштаб разбиения $\T=\{I_i\}_{i=1}^k$ — число $\lambda(\T) = \Delta_{\T} = \displaystyle\max_{1\le i\le k}$

\definition \label{1.6} Пусть $\forall\ I_i$ выбрана точка $\xi_i\in I_i$. Тогда, набор $\xi = \{\xi\}_{i=1}^k$ будем называть \textbf{отмеченными точками}

\definition \label{1.7} Размеченное разбиение — пара $(\T, \xi)$

\subsection{Интегральная сумма Римана. Интегрируемость по Риману}
Пусть $I$ — невырожденный, замкнутый брус, функция $f: I\rightarrow \R$ определена на $I$

\definition \label{1.8} Интегральная сумма Римана функции $f$ на $(\T, \xi)$ — величина
$$\sigma(f, \T, \xi) := \sum_{i=1}^kf(\xi_i)\cdot|I_i|$$

\definition \label{1.9} Функция $f$ интегрируема (по Риману) на замкнутом брусе $I$ ($f:I\rightarrow\R$), если 
\begin{equation*}
\begin{aligned}
    \exists A\in\R: \forall \varepsilon > 0\, \exists \delta > 0: \forall(\T, \xi): \Delta_{\T} < \delta:\\
    |\sigma(f, \T, \xi)| - A| < \varepsilon
\end{aligned}
\end{equation*}
Тогда 
$$A = \int_If(x)dx = \underset{I}{\int\ldots\int}f(x_1, \ldots, x_n)dx_1\ldots dx_n$$
Обозначение: $f\in\mathcal{R}(I)$

\subsection{Пример константной функции}
Пуусть у нас есть функция $f = \text{const}$
\begin{equation*}
\begin{aligned}
    \forall(\T, \xi):\ \sigma(f, \T, \xi)&= \sum_{i = 1}^k \text{const}\cdot|I_i|\\
    &= \text{const}\cdot|I| \Longrightarrow \int_I f(x)\d{x} = \text{const}\cdot|I|
    \end{aligned}
\end{equation*}

\subsection{Неинтегрируемая функция}
Имеется брус $I = [0, 1]^n$, а также определена функция, такая что
\begin{equation*}
    f = \begin{cases}
        1,& \forall i = \overline{1,\ldots, n}\,\, x_i\in \mathbb{Q}\\
        0,&\text{иначе}
    \end{cases}
\end{equation*}

\proof $\forall \T$ можно выбрать $\xi_i\in \mathbb{Q}$, тогда для такой пары $(\T, \overline{\xi})$:
\begin{equation*}
    \sigma(f, \T, \overline{\xi}) = \sum_{i=1}^k1\cdot|I_i| = |I| = 1
\end{equation*}

В то же время, $\forall \T$ можно выбрать $\xi_i\notin \mathbb{Q}$, тогда для такой пары $(\T, \hat{\xi})$:
\begin{equation*}
    \sigma(f, \T, \hat{\xi}) = \sum_{i=1}^k0\cdot|I_i| = 0 \Longrightarrow f\notin\mathcal{R}(I)
\end{equation*}

\subsection{Вычисление многомерного интеграла}
Вычислите интеграл
$$\iint\limits_{\substack{0\leqslant x\leqslant 1\\ 0\leqslant y\leqslant 1}}xy\d{x}\d{y}$$
рассматривая его как представление интегральной суммы при сеточном разбиении квадрата $$I = [0, 1]\times[0, 1]$$ на ячейки — квадраты со сторонами, длины которых равны $\frac{1}{n}$, выбирая в качестве точек $\xi_i$ верхние правые вершины ячеек

\begin{minipage}{0.5\textwidth}
Имеется функция $f = xy,\ |I| =\displaystyle\frac{1}{n^2}$
\begin{equation*}
    \begin{aligned}
        \sigma(f, \T, \xi) &= \sum_{i=1}^n\sum_{j=1}^n\frac{i}{n}\cdot\frac{j}{n}\cdot\frac{1}{n^2}\\
        &= \frac{1}{n^4}\sum_{i=1}^n\sum_{j=1}^n i\cdot j\\
        &= \frac{1}{n^4}\sum_{i=1}^ni\sum_{j=1}^nj\\
        &= \frac{n(n+1)}{n^4}\sum_{i=1}^ni\\
        &= \frac{n^2(n+1)^2}{4n^4}
        % \underset{n\to\infty}{\longrightarrow}\frac{1}{4}
    \end{aligned}
\end{equation*}
Заметим, что $\lim\limits_{n\rightarrow\infty}\displaystyle\frac{n^2(n+1)^2}{4n^4}=\frac{1}{4}$
\end{minipage}
\begin{minipage}{0.5\textwidth}
$$
    \begin{tikzpicture}[scale=3]
        \draw[step=0.25cm, gray, very thin] (0,0) grid (2,2);

        \draw[thick] (0,0) rectangle (2,2);
        
        \node at (-0.1, -0.1) {$0$};
        \node at (1, -0.3) {\text{$n$ штук}};
        \node at (2.1, -0.1) {$1$};
        \node at (-0.1, 2) {$1$};

        \foreach \i in {0.25, 0.5, ..., 2} {
        \foreach \j in {0.25, 0.5, ..., 2} {
            \fill (\i,\j) circle (1pt);
        }
    }

    \end{tikzpicture}
$$
\end{minipage}

\subsection{Свойства кратных интегралов}
\begin{enumerate}
    \item \textbf{Линейность.}
    \begin{equation*}
        f, g \in R(I) \implies \forall \alpha, \beta \in \R \quad(\alpha f + \beta g)\in R(I)
    \end{equation*}
    И верно, что:
    \begin{equation*}
            \int_I(\alpha f + \beta g)dx = \alpha\int_I fdx + \beta\int_Igdx
    \end{equation*}
\proof 
\begin{enumerate}
    \item
\begin{equation*}
\begin{aligned}
    f\in R(I): \quad \forall \varepsilon > 0 \, \exists\delta_1>0: \Delta_{\T} < \delta_1 \\
    |\sigma(f, \T, \Xi)  - \int_Ifdx| =: |\sigma_f - A_f| < \frac{\varepsilon}{|\alpha|+|\beta|+1}
\end{aligned}
\end{equation*}
\item По определению:
\begin{equation*}
    \begin{aligned}
        g\in R(I): \quad \forall \varepsilon > 0 \, \exists\delta_2>0: \Delta_{\T} < \delta_2 \\
|\sigma(g, \T, \Xi)  - \int_Igdx| =: |\sigma_g - A_g| < \frac{\varepsilon}{|\alpha|+|\beta|+1}
    \end{aligned}
\end{equation*}
\item Пусть $\delta = \min\{\delta_1, \delta_2\}$. Тогда (a) и (b) верно для $\delta \implies$
\begin{equation*}
    \begin{aligned}
        |\sigma_{\alpha f+\beta g} - A_{\alpha f+ \beta g}| = |\alpha\sigma_f + \beta\sigma_g - \alpha A_f - \beta A_g| \le \\
        |\alpha||\sigma_f - A_f| + |\beta||\sigma_g-A_g| < (|\alpha| + |\beta|) \frac{\varepsilon}{|\alpha|+|\beta|+1} < \varepsilon
    \end{aligned}
\end{equation*}
\end{enumerate}
\QED

\item \textbf{Монотонность}
\begin{equation*}
    f, g\in R(I); f|_I\le g|_I \implies \int_Ifdx \le \int_Igdx
\end{equation*}
\proof
    \begin{equation*}
        f\in R(I) \implies \exists A_f\in \R : |\sigma_f - A_f| < \ve\, (\forall \ve > 0 \exists\delta: \forall(\T, \Xi): \Delta_{\T} < \delta)
    \end{equation*}
    Аналогично для $g\in R(I)$, тогда:
    \begin{equation*}
    \begin{aligned}
        A_f - \ve < \sigma_f \le \sigma_g < A_g + \ve \implies\\
        A_f < A_g + 2\ve 
    \end{aligned}
    \end{equation*}
    Что верно для $\forall \ve > 0$, даже при $\ve \to 0 \implies A_f \le A_g$
\QED
\item \textbf{Оценка интеграла (сверху)}
\begin{equation*}
    f\in R(I) \implies \left|\int_Ifdx\right| \le \underset{I}{\sup}|f||I|
\end{equation*}
\proof
По необходимому условию для интегрируемости функции (см. ниже)
\begin{equation*}
    f\in R(I) \implies f \text{ Ограничена на } I \implies
\end{equation*}

\begin{equation*}
    \begin{aligned}
        -\sup_I|f| \le f \le \sup_I|f| \implies\\
        -\int_I\sup|f|dx \le \int_Ifdx\le \int_I\sup|f|dx = \sup_I|f||I| \implies\\
        -\sup_I|f||I| \le A_f \le \sup_I|f||I|
    \end{aligned}
\end{equation*}
\QED
\end{enumerate}

\subsection{\theorem Необходимое условие интегрирования.}
Пусть $I$ - замкнутый брус. 
\begin{equation*}
    f\in R(I) \implies f \text{ Ограничена на } I
\end{equation*}

\proof От противного.
\begin{enumerate}
    \item \begin{equation*}
        \begin{aligned}
            f\in R(I) \implies \exists A_f\in\R : \forall\ve > 0 \, \exists\delta > 0: \forall(\T, \Xi): \Delta_{\T} < \delta\\
            |\sigma_f-A_f| < \ve \implies \\
        \end{aligned}
    \end{equation*}
    Для $\ve = 1$ это тоже верно, поэтому:
    \begin{equation*}
        A_f-1<\sigma_f<A_f+1 \implies \sigma_f - \text{ ограничена}
    \end{equation*}
    \item Пусть $f$ - неограничена на $I$, но $f\in R(I) \implies \forall\T = \{I_i\}_i=1^K \,\, \exists i_0: \, f$ неограничена на $I_{i_0}$.\\
    Тогда можно представить так: 
    \begin{equation*}
        \sigma_f = \sum_{i\ne i_0}f(\xi_i)|I_i| + f(\xi_{i_0})|I_{i_0}| \implies
    \end{equation*}
    можно выбирать $\xi_{i_0}$ на $I_{i_0}$ так, что $\sigma_f$ будет сколь угодно велика/мала $\implies$ \textbf{противоречие}, поэтому:
\end{enumerate}
\begin{equation*}
    f\in R(I) \implies f \text{ Ограничена на } I
\end{equation*}
\QED

\section{Лебегова мера}
\subsection{Множество меры 0 по Лебегу}
\definition Мн-во $M\subset\R^n$ будем называть мн-вом меры 0 по Лебегу, если $\exists$не более чем счетный набор замкнутых брусов $\{I_i\}:$
\begin{enumerate}
    \item $M\subset \displaystyle\bigcup_iI_i$ и 
    \item $\displaystyle\sum_i|I_i| < \ve\,\, \forall \ve < 0$
\end{enumerate}

\textbf{Пример:} $x_0\in\R^n$ - мн-во меры нуль по Лебегу в $\R^n$\\
\proof $x_0 = (x_{01}, \ldots, x_{0n})$
Покроем точку замкнутым брусом (тут Даня должен вставить крутую картинку как квадрат накладывается на точку) $I = [x_{01}-d, x_{01}+d] \times\ldots\times[x_{0n}-d, x_{0n}+d \implies$\\
$$\forall \ve > 0\,\,\exists I: |I| = (2d)^n<\ve \implies d < \frac{\sqrt[n]{\ve}}{2} \implies$$
Точка является мн-вом меры нуль по Лебегу

\subsection{Свойства мн-ва меры 0 по Лебегу}
\begin{enumerate}
    \item Если брусы в определении сделать открытыми, то определение остается верным.\\
    \proof Пусть $\{I_i\}$ - открытые брусы  $M\subset \displaystyle\bigcup_iI_i$.\\
    Пусть $\{\bar I_i\}$ - замкнутые брусы $I_i$.
    \begin{equation*}
        \begin{aligned}
            M\subset\bigcup_iI_i \subset\bigcup_i\bar I_i, \, |I_i| = |\bar I_i|
        \end{aligned}
    \end{equation*}
    Если
    \begin{equation*}
        \forall\ve\, \exists\{I_i\}: M \subset \bigcup_iI_i: \sum_i|I_i|<\ve
    \end{equation*}
    то
    \begin{equation*}
        \forall\ve\, \exists\{\bar I_i\}: M \subset \bigcup_i\bar I_i: \sum_i|\bar I_i|<\ve
    \end{equation*}
    \item Пусть $\{I_i\}$ - замкнутые брусы
    \begin{equation*}
        I_i = [a^i_1, b^i_1]\times\ldots\times[a^i_n, b^i_n] \quad V_i = \sum_i|I_i|<\frac{\ve}{2^n}
    \end{equation*}
    Пусть 
    \begin{equation*}
        D_i = \left(\frac{a_1^i+b_1^i}{2} - (b_1^i-a_1^i) ; \frac{a_1^i + b_1^i}{2} + (b_1^i - a_1^i)\right) \times \ldots\times \left(\frac{a_n^i+b_n^i}{2} - (b_n^i-a_n^i) ; \frac{a_n^i + b_n^i}{2} + (b_n^i - a_n^i)\right)
    \end{equation*}
    $\implies V_2 = \displaystyle\sum_i|D_i| = 2^nV_1 < \ve$
    \QED
    
\end{enumerate}
\end{document}