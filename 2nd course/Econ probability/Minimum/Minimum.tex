\documentclass{article}
\usepackage{header} % Required for inserting images
% \newcommand{\p}[1]{$\mathbb{P}(#1)$}


\title{\LARGE{Теория вероятности и математическая статистика—1}\\
Теоретический и задачный минимумы\\
ФЭН НИУ ВШЭ}
\author{Винер Даниил  \href{https://t.me/danya_vin}{@danya\_vin}}
\date{Версия от \today}

\begin{document}
\maketitle
\tableofcontents
\newpage
\setlength{\parindent}{15pt}
\setlength{\parskip}{2mm}
\section{Теоретический минимум}
\subsection{Сформулируйте классическое определение вероятности}
Имеет место, когда исходы равновероятны

\definition
$$P(A)=\displaystyle\frac{|A|}{|\Omega|}$$

\definition $P(A)=\sum_{\omega_i\in A}p(\omega_i)$
\subsection{Выпишите формулу условной вероятности}
$P(A|B)=\displaystyle\frac{P(A\cap B)}{P(B)}\ \forall B: P(B)>0$

\subsection{Дайте определение независимости (попарной и в совокупности) для $n$ случайных событий}
\definition События $A\text{ и }B$ называются \textbf{попарно независимыми}, если:
\begin{equation*}
    \begin{aligned}
        P(A\cap B)&=P(A)\cdot P(B)\\
        P(A|B)P(B)&=P(A)\cdot P(B)\text{ — вытекает интуитивное определение}
    \end{aligned}
\end{equation*}

\definition События $A_1,\ldots, A_n$ независимы в совокупности, если:
\begin{equation*}
    \begin{aligned}
    \forall i_1<\ldots<i_k<\ldots<i_n\ \forall k=1,\ldots,n:\\ P(A_{i_1}\cap A_{i_2}\cap\ldots\cap A_{i_k})=P(A_{i_1})\cdot P(A_{i_2})\cdot \ldots\cdot P(A_{i_k})
    \end{aligned}
\end{equation*}

\comment  Для $A_1,A_2,A_3$:
\begin{equation*}
    \begin{aligned}
        P(A_1\cap A_2)&=P(A_1)\cdot P(A_2)\\
        P(A_2\cap A_3)&=P(A_2)\cdot P(A_3)\\
        P(A_1\cap A_3)&=P(A_1)\cdot P(A_3)\\
        P(A_1\cap A_2\cap A_3)&=P(A_1)\cdot P(A_2)\cdot P(A_3)
    \end{aligned}
\end{equation*}

\subsection{Выпишите формулу полной вероятности, указав условия её применимости}
Пусть $\{H_i\}$ — полная группа несовместных событий (разбиение $\Omega$)

Должны быть выполнены такие свойства:
\begin{itemize}
    \item $H_i\cap H_j=\varnothing\ \forall i\ne j$ — несовместность\\[1mm]
    \item $\displaystyle\bigcup_{i=1}^{n}H_i=\Omega$ — полнота
\end{itemize}

\theorem Тогда, $P(A)=\sum_{i=1}^{n}P(A|H_i)P(H_i)$

\proof
\begin{equation*}
    \begin{aligned}
        P(A)&=P\left(\bigcup_{i=1}^{n}(A\cap H_i)\right)\\
        &=\sum_{i=1}^{n}P(A\cap H_i)\\
        &=\sum_{i=1}^{n} P(A|H_i)\cdot P(H_i)
    \end{aligned}
\end{equation*}\qed

\subsection{Выпишите формулу Байеса, указав условия её применимости}
Пусть $H_1, H_2, \ldots$ — полная группа событий, и $A$ — некоторое событие, вероятность которого положительна. Тогда условная вероятность того, что имело место событие $H_k$, если в резулътате эксперимента наблюдалось событие $A$, может быть вычислена по формуле
\begin{equation*}
    \begin{aligned}
        P(H_k|A)&=\frac{P(A|H_k)\cdot P(H_k)}{P(A)}\\
        &=\frac{P(H_k\cap A)}{P(A)}\\
        &=\frac{P(A|H_k)\cdot P(H_k)}{\sum_{i=1}^{n} P(A|H_i)P(H_i)}
    \end{aligned}
\end{equation*}

\newpage
\section{Задачный минимум}
\subsection{$P(A)=0.3,P(B)=0.4,P(A\cap B)=0.1$}
\begin{enumerate}
    \item[\textbf{a)}] Найдите $P(A|B)$
    
    $P(A|B)=\displaystyle\frac{P(A\cap B)}{P(B)}=\frac{0.1}{0.4}=0.25$
    \item[\textbf{b)}] Найдите $P(A\cup B)$
    
    $P(A\cup B)=P(A)+P(B)-P(A\cap B)=0.3+0.4-0.1=0.6$
    \item[\textbf{c)}] Являются ли события $A$ и $B$ независимыми?

    \definition События $A$ и $B$ называются независимыми, если $P(A\cap B)=P(A)\cdot P(B)$

    \definition События $A$ и $B$ называются несовместными, если $A\cap B=\varnothing$

    Пусть $P(A)\ne0,P(B)\ne0$. Тогда, $A$ и $B$ несовместны, то $A$ и $B$ зависимы

    $$0=P(A\cap B)=P(A)\cdot P(B)\ne0\Longrightarrow A \text{ и } B \text{ зависимы}$$
\end{enumerate}

\subsection{Карлсон выложил кубиками слово КОМБИНАТОРИКА...}
\subsubsection*{Способ №1 (С помощью формулы умножения вероятностей)}
$P(A_1\cap \ldots \cap A_n)=P(A_1)\cdot P(A_2|A_1)\cdot P(A_3|A_1\cap A_2)\cdot\ldots\cdot P(A_n|A_1\cap\ldots\cap A_{n-1})$



Пусть имеются такие события: \begin{equation*}
    \begin{aligned}
        A_1&:=\{\text{первая буква — К}\}\\
        A_2&:=\{\text{вторая буква — О}\}\\
        A_3&:=\{\text{третья буква — Р}\}\\
        A_4&:=\{\text{четвертая буква — Т}\}
    \end{aligned}
\end{equation*}

Тогда, искомая вероятность:
\begin{equation*}
\begin{aligned}
    P(A_1\cap A_2\cap A_3\cap A_4)&=P(A_1)\cdot P(A_2|A_1)\cdot P(A_3|A_1\cap A_2)\cdot P(A_4|A_1\cap A_2\cap A_3)\\
    &=\frac{2}{13}\cdot\frac{2}{12}\cdot\frac{1}{11}\cdot\frac{1}{10}\\
    &=\frac{1}{4290}
    \end{aligned}
\end{equation*}

\subsubsection*{Способ №2 (комбинаторный)}
$P(A)=\displaystyle\frac{|A|}{|\Omega|},\ \Omega=\{(a_1,a_2,a_3,a_4):a_1\in L, a_2\in L, a_3\in L, a_4\in L, a_i\ne a_j\text{ при }i\ne j\}$

$|\Omega|=\displaystyle\frac{13!}{9!}=17160$

$A=\{(K_1,O_1,P_1,T_1),(K_2,O_1,P_1,T_1),(K_1,O_2,P_1,T_1),(K_2,O_2,P_1,T_1)\}\longrightarrow$ 4 исхода

Индекс у букв означают какой по счету встретилась буква в слове <<КОМБИНАТОРИКА>>

Тогда, искомая вероятность$=\displaystyle\frac{|A|}{|\Omega|}=\frac{4}{17160}=\frac{1}{4290}$

\subsection{В первой урне 7 белых и 3 черных шара, во второй — 8 белых и 4 черных шара, в третьей — 2 белых и 13 черных шаров}
$D_i:=\{\text{выбираем $i$-ю урну}\}$, где $i=1,2,3$ — разбиение $\Omega$

Заметим, что урну мы выбираем равновероятно, то есть $P(D_1)=P(D_2)=P(D_3)=\displaystyle\frac{1}{3}$
\begin{enumerate}
    \item[\textbf{a)}] Вычислите вероятность того, что шар, взятый наугад из выбранной урны, окажется белым
    
    \textbf{Формуа полной вероятности}
    $$P(A)=P(A|D_1)\cdot P(D_1)+\ldots+P(A|D_n)\cdot P(D_n)$$
    В нашем случае, формула будет иметь вид
    $$P(A)=P(A|D_1)\cdot P(D_1)+P(A|D_2)\cdot P(D_2)+P(A|D_3)\cdot P(D_3)$$
    $A:=\{\text{шар оказался белым}\}$

    Заметим, что $P(A|D_1)=\frac{7}{10},P(A|D_2)=\frac{2}{3},P(A|D_3)=\frac{2}{15}$, тогда
    \begin{equation*}
        \begin{aligned}
            P(A)&=P(A|D_1)\cdot P(D_1)+P(A|D_2)\cdot P(D_2)+P(A|D_3)\cdot P(D_3)\\
            &=\frac{7}{10}\cdot\frac{1}{3}+\frac{2}{3}\cdot\frac{1}{3}+\frac{2}{15}\cdot\frac{1}{3}\\
            &=\frac{1}{2}
        \end{aligned}
    \end{equation*}
    \item[\textbf{b)}] $P(D_1|A)=\displaystyle\frac{P(A|D_1)\cdot P(D_1)}{P(A|D_1)P(D_1)+P(A|D_2)P(D_2)+P(A|D_3)P(D_3)}=\frac{7}{15}$
\end{enumerate}

\subsection{В операционном отделе банка работает 80\% опытных сотрудников и 20\% неопытных}
Обозначим сотрудников так:
\begin{equation*}
    \begin{aligned}
        D_1&:=\{\text{\textbf{опытный} сотрудник}\}\\
        D_2&:=\{\text{\textbf{неопытный} сотрудник}\}
    \end{aligned}
\end{equation*}

Пусть $A:=\{\text{совершена ошибка}\}$

Тогда, условия задачи можно записать так:
\begin{equation*}
    \begin{aligned}
        P(A|D_1)&=0.01\\
        P(A|D_2)&=0.1
    \end{aligned}
\end{equation*}

\begin{enumerate}
    \item[\textbf{a)}] $P(A)=P(A|D_1)\cdot P(D_1)+P(A|D_2)\cdot P(D_2)=0.01\cdot0.8+0.1\cdot0.2=0.028$
    \item[\textbf{b)}] $P(D_2|A)=\displaystyle\frac{P(A|D_2)\cdot P(D_2)}{P(A)}=0.714$

    Заметим, что события $(D_2|A)$ и $(D_1|A)$ образуют полную группу вероятностей, то есть $$P(D_2|A)+P(D_1|A)=1\Longrightarrow P(D_1|A)=0.286$$
\end{enumerate}

\subsection{Пусть случайная величина $X$ имеет таблицу распределения}
\begin{table}[h]
    \begin{tabular}{|c|c|c|c|}
        \hline
        $x$ & -1 & 0 & 1 \\
        \hline
        $\mathbb{P}(\{X = x\})$ & $0.25$ & $c$ & $0.25$ \\
        \hline
    \end{tabular}
\end{table}

\begin{itemize}
    \item[\textbf{а)}] $\Omega = \{X=-1\} + \{X = 0\} + \{X = 1\} \text{ и } 1 = \prob{\{X=-1\}} + \prob{\{X=0\}} + \prob{\{X=1\}} \implies c = 0.5$
    \item[\textbf{б)}]  $\mathbb{P}\{X\geqslant0\} = \mathbb{P}(\{X=0\}\sqcup\{X=1\}) = \mathbb{P}(\{X=0\}) + \mathbb{P}(\{X=1\}) = 0.75$
    \item[\textbf{в)}]  $\mathbb{P}(\{X<-3\}) = 0$, т.к. $\Omega$ — дискретное простраство, или же $\{X<-3\} = \{\omega\in \Omega: X(\omega)<-3\}$
    \item[\textbf{г)}]  $\mathbb{P}(\{X\in [-0.5;0.5]\}) = \mathbb{P}(\{X=0\}) = 0.5$, т.к. $\Omega$ — дискретное простраство
\end{itemize}

\subsection{Пусть случайная величина $X$ имеет таблицу распределения}


\begin{itemize}
    \item[\textbf{а)}] Аналогично предыдущей задаче — $c=0.5$
    % $F_X(-100) = \mathbb{P}(\{\omega\in\Omega:\xi(\omega)\leqslant -100\}) = 0$

    % $$\begin{aligned}
    %     F_X(0.77) &= \mathbb{P}(\{\omega\in\Omega:\xi(\omega)\leqslant 0.77\})\\
    %     &=\mathbb{P}(\{X\leqslant 0.77\})\\
    %     &= \mathbb{P}(\{X=-1\}\sqcup\{X=0\}\\
    %     &= \mathbb{P}(\{X=-1\}) \mathbb{P}(\{X = 0\}\\
    %     &= 0.75
    % \end{aligned}$$
    \item[\textbf{б)}] $\mathbb{E}[X] = -1\cdot0.25 + 0\cdot0.5+1\cdot0.25 = 0$
    \item[\textbf{в)}] $\mathbb{E}[X^2] = (-1)^2\cdot0.25 + 0\cdot0.5 + 1\cdot0.25 = 0.5$

    $\mathbb{E}[\sin{(X)}] = \sin(-1)\cdot0.25 + \sin(0)\cdot0.5 + \sin(1)\cdot0.25$
    \item[\textbf{г)}] $\dispersia{X} \equiv \dispersia{X} := \mathbb{E}[(X-\mathbb{E}[X])^2] = \mathbb{E}[X^2]- \mathbb{E}^2[X]$
    \item[\textbf{д)}] $\mathbb{E}[|X|] = |-1|\cdot0.25 + |0|\cdot0.25 + |1|\cdot0.25 = 0.5$

\end{itemize}

\begin{table}[h]
    \begin{tabular}{|c|c|c|c|}
        \hline
        $x$ & -1 & 0 & 1 \\
        \hline
        $\mathbb{P}(\{\xi = x\})$ & $0.25$ & $c$ & $0.25$ \\
        \hline
    \end{tabular}
\end{table}

\subsection{Пусть случайная величина $X$ имеет биномиальное распределение с параметрами $n = 4$ и $p = 0.75$}
$X\sim\text{Bi}(n=4,p=\frac{3}{4})$. Напомним, что $\prob{X=k}=C_4^k (\frac{3}{4})^k(\frac{1}{4})^{4-k}$
\begin{itemize}
    \item[\textbf{a)}] $\prob{X=0}=C_4^0 \left(\displaystyle\frac{3}{4}\right)^0\left(\displaystyle\frac{1}{4}\right)^{4}=\left(\displaystyle\frac{1}{4}\right)^4$
    \item[\textbf{б)}] $\prob{X>0}=1-\prob{X=0}=1-\left(\displaystyle\frac{1}{4}\right)^4$
    \item[\textbf{в)}] $\prob{X<0}=0$, так как количество успехов в биномиальном распределении $\geqslant0$
    \item[\textbf{г)}] $\matwait{X}=n\cdot p=4\cdot\frac{3}{4}=3$
    \item[\textbf{д)}] $\dispersia{X}=np(1-p)=\frac{3}{4}$
    \item[\textbf{е)}] Нужно посчитать наиболее вероятную величину. Всего есть 5 значений — 5 возможных успешных исходов
    
    $\prob{X=0}=\left(\displaystyle\frac{1}{4}\right)^4$
    
    $\prob{X=1}=C_4^1\cdot\displaystyle\frac{3}{4}\cdot\left(\frac{1}{4}\right)^3$

    $\prob{X=2}=C_4^2\cdot\displaystyle\left(\frac{3}{4}\right)^2\cdot\left(\frac{1}{4}\right)^2$

    $\prob{X=3}=C_4^3\cdot\displaystyle\left(\frac{3}{4}\right)^3\cdot\left(\frac{1}{4}\right)^1$

    $\prob{X=4}=C_4^4\cdot\displaystyle\left(\frac{3}{4}\right)^4\cdot\left(\frac{1}{4}\right)^0$

\end{itemize}
\subsection{Пусть случайная величина $X$ имеет распределение Пуассона с параметром $\lambda= 100$}
Имеется случайная величина $X\sim\text{Pois}(\lambda=100)$
\begin{enumerate}
    \item[\textbf{a)}] $\prob{\{X=0\}}=\displaystyle\frac{\lambda^0}{0!}e^{-\lambda}=e^{-\lambda}=e^{-100}$
    \item[\textbf{б)}] $\prob{\{X>0\}}=1-\prob{\{x=0\}}=1-e^{-100}$
    \item[\textbf{в)}] $\prob{\{X<0\}}=\prob{\varnothing}=0$
    \item[\textbf{г)}] По определению, $\mathbb{E}\left[X\right]=\lambda$. Докажем
    \begin{equation*}
        \begin{aligned}
            \matwait{X}&=\sum_{k=0}^{\infty} k\cdot\prob{\{x=k\}}\\
            &=\sum_{k=0}^{\infty} k\frac{\lambda^k}{k!}e^{-\lambda}\\
            &=\sum_{k=1}^{\infty} \frac{\lambda^k}{(k-1)!}e^{-\lambda}\\
            &=\lambda e^{-\lambda}\sum_{k=1}^{\infty}\frac{\lambda^{k-1}}{(k-1)!}\\
            &=\left(\sum_{l=0}^{\infty}\frac{\lambda^l}{l!}\right)\lambda e^{-\lambda}\\
            &=\lambda
        \end{aligned}
    \end{equation*}
    \item[\textbf{д)}] Для того, чтобы посчитать дисперсию $X$ сначала посчитаем мат.ожидание $X^2$, а для этого посчитаем $\matwait{X(X-1)}$:
    \begin{equation*}
        \begin{aligned}
            \matwait{X(X-1)}&=\sum_{k=0}^{\infty} k(k-1)\prob{\{x=k\}}\\
            &=\sum_{k=2}^{\infty} k(k-1)\frac{\lambda^k}{k!}e^{-\lambda}\\
            &=\lambda^2e^{-\lambda}\sum_{k=2}^{\infty}\frac{\lambda^{k-2}}{(k-2)!}e^{-\lambda}\\
            &=\lambda^2e^{-\lambda}\sum_{l=0}^{\infty}\frac{\lambda^{l}}{l!}\\
            &=\lambda
        \end{aligned}
    \end{equation*}
    Тогда, $\lambda^2=\matwait{X(X-1)}=\matwait{X^2}-\matwait{X}\Longrightarrow\matwait{X^2}=\lambda+\lambda^2$
    
    Теперь можем выразить дисперсию через известное равенство:
    \begin{equation*}
        \dispersia{X}=\matwait{X^2}-\left(\matwait{X}\right)^2=\lambda+\lambda^2-\lambda^2=\lambda
    \end{equation*}
    \item[\textbf{e)}] Предположим, что $X=k$ и есть наиболее вероятное значение, принимаемое $X$. При этом, $k\in\{0,1,2,\ldots\}$. Так как $k$ — дискретная, то дифференцированием мы воспользоваться не можем, тогда посчитаем $\displaystyle\frac{\prob{\{X=k+1\}}}{\prob{\{X=k\}}}$:
    \begin{equation*}
        \begin{aligned}
            \frac{\prob{\{X=k+1\}}}{\prob{\{X=k\}}}&=\frac{\frac{\lambda^{k+1}}{(k+1)!}e^{-\lambda}}{\frac{\lambda^{k}}{k!}e^{-\lambda}}\\
            &=\frac{\lambda}{k+1}\\
            &=\frac{100}{k+1}
        \end{aligned}
    \end{equation*}
    Теперь проанализируем при каких $k$ это отношение будет больше, меньше или равно $1$:
    \begin{itemize}
        \item $\displaystyle\frac{100}{k+1}>1\Longrightarrow k <99$
        \item $\displaystyle\frac{100}{k+1}<1\Longrightarrow k >99$
        \item $\displaystyle\frac{100}{k+1}=1\Longrightarrow k =99$
    \end{itemize}
    Значит, $99$ и $100$ — наиболее вероятные значения, принимаемые случайной величиной $X$
    $$
    \begin{tikzpicture}
        \draw[black, ->] (0,0) -- (0,2.5) node[black, left] {$\prob{\{X=k\}}$};
        \draw[black, ->] (0,0) -- (7,0) node[black, right] {$k$};
        \draw[black, thick] (0,0) -- (2,1);
        \draw[black, very thick] (2,1) -- (4,1);
        \draw[black, thick] (4,1) -- (6,0);
        \draw[black, dashed] (2,0) node[black, below] {$99$} -- (2,1);
        \draw[black, dashed] (4,0) node[black, below] {$100$} -- (4,1);
    \end{tikzpicture}
    $$
\end{enumerate}
\subsection{В лифт 10-этажного дома на первом этаже вошли 5 человек}
\begin{enumerate}
    \item[\textbf{а)}] Пусть $\xi_i=\begin{cases}
        1,&\text{если $i$-й \textit{пассажир} вышел на шестом этаже}\\
        0,&\text{иначе}
    \end{cases}$. При этом $i\in\{1,2,3,4,5\}$

    Тогда, $\xi=\xi_1+\ldots+\xi_5$ — число \textit{пассажиров}, которые вышли на шестом этаже

    Заметим, что $\xi_1,\ldots,\xi_5$ — независимые, а также $\xi_i\sim\text{Be}\left(p=\frac{1}{9}\right)$. Тогда, $\xi\sim\text{Bi}\left(n=5,p=\frac{1}{9}\right)$

    $\prob{\{\xi>0\}}=1-\prob{\{\xi=0\}}=1-\left(\frac{8}{9}\right)^5$
    \item[\textbf{б)}] $\prob{\{\xi=0\}}=C_n^k p^k q^{n-k}=C_5^0\left(\frac{1}{9}\right)^0\left(\frac{8}{9}\right)^5=\left(\frac{8}{9}\right)^5$
    \item[\textbf{в)}] Пусть $\eta_i=\begin{cases}
        1,&\text{если $i$-й \textit{пассажир} вышел на 6 этаже или выше}\\
        0,&\text{иначе}
    \end{cases}$. При этом $i\in\{1,2,3,4,5\}$

    Тогда, $\eta=\eta_1+\ldots+\eta_5$ — число \textit{пассажиров}, которые вышли на шестом этаже и выше

    Заметим, что $\eta_1,\ldots,\eta_5$ — независимые, а также $\eta_i\sim\text{Be}\left(p=\frac{5}{9}\right)$. Тогда, $\eta\sim\text{Bi}\left(n=5,p_1=\frac{5}{9}\right)$

    $\prob{\{\eta=5\}}=C_5^5\cdot p_1^5\cdot q^0=\left(\frac{5}{9}\right)^5$
\end{enumerate}
\subsection{При работе некоторого устройства время от времени возникают сбои}
$\xi_i\sim\text{Pois}(\lambda=3)$ — число сбоев за $i$-е сутки
\begin{enumerate}
    \item[\textbf{а)}] \begin{equation*}
        \begin{aligned}
            \prob{\{\xi_i>0\}}&=1-\prob{\{\xi_i=0\}}\\
            &=1-\frac{\lambda^0}{0!}e^{-\lambda}\\
            &=1-e^{-3}
        \end{aligned}
    \end{equation*}
    \item[\textbf{б)}] Требуется вычислить вероятность того, что за двое суток не произойдет ни одного сбоя. То есть нужно найти вероятность двух событий: $\{\xi_1=0\}$ и $\{\xi_2=0\}$. Заметим, что эти события независимы. Формально:
    \begin{equation*}
        \begin{aligned}
            \prob{\{\xi_1=0\}\cap\{\xi_2=0\}}&=\prob{\{\xi_1=0\}}\cdot\prob{\{\xi_2=0\}}\\
            &=e^{-3}\cdot e^{-3}
        \end{aligned}
    \end{equation*}
\end{enumerate}



\end{document}
