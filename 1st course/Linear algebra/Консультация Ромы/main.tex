\documentclass[a4paper]{article}
\usepackage[utf8]{inputenc}
\usepackage[T2A]{fontenc}
\usepackage[english, russian]{babel}
\usepackage[left=25mm, top=20mm, right=25mm, bottom=30mm, nohead, nofoot]{geometry}
\usepackage{amsmath, amsfonts, amssymb} % математический пакет
\usepackage {fancybox, fancyhdr}
\usepackage{python}
\pagestyle{fancy}
\fancyhf{}
\fancyhead[R]{\LaTeX\ by Viner Daniil BEDA232}
\fancyfoot [C] {\thepage}
\fancyhead[L]{\textbf{Linear algebra and geometry}}
\setcounter {page}{1}
\headsep=10mm
\usepackage{xcolor}
\usepackage{cancel}
\usepackage{ulem}
\usepackage{hyperref}
\usepackage{epigraph}
\hypersetup{colorlinks=true, allcolors= [RGB]{010 090 200}} % цвет ссылок
\usepackage {setspace}
\usepackage[pdftex]{graphicx}
\usepackage{ dsfont }
\usepackage{array}
\setcounter{MaxMatrixCols}{20}
\usepackage{mathbbol}
\usepackage{minted}
\usepackage{enumerate}
\usepackage{listings}
% \usepackage{breqn}
\usepackage{indentfirst}
\usepackage{color}
\usepackage{mathtools} 
\usepackage{extarrows} 
\usepackage{tcolorbox}
\definecolor{dkgreen}{rgb}{0,0.6,0}
\definecolor{gray}{rgb}{0.5,0.5,0.5}
\definecolor{mauve}{rgb}{0.58,0,0.82}
\lstset{frame=tb,
  language=C++,
  aboveskip=3mm,
  belowskip=3mm,
  showstringspaces=false,
  columns=flexible,
  basicstyle={\small\ttfamily},
  numbers=none,
  numberstyle=\tiny\color{gray},
  keywordstyle=\color{blue},
  commentstyle=\color{dkgreen},
  stringstyle=\color{mauve},
  breaklines=true,
  breakatwhitespace=true,
  tabsize=3
}
% «»

\DeclareRobustCommand{\divby}{%
  \mathrel{\text{\vbox{\baselineskip.65ex\lineskiplimit0pt\hbox{.}\hbox{.}\hbox{.}}}}%
}
\DeclareSymbolFontAlphabet{\mathbb}{AMSb}
\DeclareSymbolFontAlphabet{\mathbbl}{bbold}
\newcommand{\e}{\mathbbl{e}}
\newcommand{\m}[1]{\mathbf{1}}
\newcommand{\qed}{\hfill$\square$}
\newcommand{\dfs}{\textbf{DFS }}
\newcommand{\eps}{\varepsilon}
\newcommand{\vphi}{\varphi}
\newcommand{\mat}{\text{Mat}}
\newcommand{\rk}{\text{rk }}
\definecolor{codegray}{gray}{0.9}
\newcommand{\pr}{\text{pr}}
\newcommand{\code}[1]{\colorbox{codegray}{\texttt{1}}}
\newcommand{\im}{\text{Im }}
\newcommand{\labeledmatrix}[2]{
    \begin{array}{ll}
        #1 & #2
    \end{array}
}
\newcommand{\ort}{\text{ort}}


% \pagecolor{black}
% \color{white}
\begin{document}

\begin{center}
    \LARGE{\textbf{Consultation for Exam—2 by $\mathbb{R}$oman Avdeev}}
\end{center}
\begin{tcolorbox}[colback=blue!20!white, colframe=black!100!black]
    \textbf{2022/2023. Вариант 1. Задача 1.} Определите все значения, которые может принимать размерность ядра линейного оператора $\varphi: \mathbb{R}^{4} \rightarrow \mathbb{R}^{4}$ при условии, что в пересечении ядра и образа содержится вектор $v=(1,0,-1,2)$.
\end{tcolorbox}
Пусть вектор $v=e_1$\\[2mm]
\indent Очевидно, что
$$\begin{aligned}
    \dim{\ker{\varphi}}&\geqslant1\\
    \dim{\im{\varphi}}&\geqslant1\\
    \dim{\ker{\varphi}}&+\dim{\im{\varphi}}=4
\end{aligned}$$
$\dim{\ker{\varphi}}=3$\\[2mm]
\indent Дополним $e_1$ до базиса $$\begin{aligned}
    e_2=(0,1,0,0)\\
    e_3=(0,0,1,0)\\
    e_4=(0,0,0,1)
\end{aligned}$$
Теперь в базисе $\e=(e_1,e_2,e_3,e_4)$ построим матрицу линейного отображения\\[2mm]
Рассмотрим несколько случаев
\begin{enumerate}
    \item $\dim{\ker{\varphi}}=1\Rightarrow\dim{\im{\varphi}}=3$\\[2mm]
    $A(\varphi,\e)=\begin{pmatrix}
    0&1&0&0\\
    0&0&0&0\\
    0&0&1&0\\
    0&0&0&1
\end{pmatrix}$
\item $\dim{\ker{\varphi}}=2\Rightarrow\dim{\im{\varphi}}=2$\\[2mm]
    $A(\varphi,\e)\begin{pmatrix}
    0&0&1&0\\
    0&0&0&1\\
    0&0&0&0\\
    0&0&0&0
\end{pmatrix}$
\item $\dim{\ker{\varphi}}=3\Rightarrow\dim{\im{\varphi}}=1$\\[2mm]
$A(\varphi,\e)=\begin{pmatrix}
    0&0&0&1\\
    0&0&0&0\\
    0&0&0&0\\
    0&0&0&0
\end{pmatrix}$
\end{enumerate}
\newpage

\begin{tcolorbox}[colback=blue!20!white, colframe=black!100!black]
    \textbf{2022/2023. Вариант 1. Задача 2.} Приведите пример неопределённой квадратичной формы $Q: \mathbb{R}^{3} \rightarrow \mathbb{R}$, принимающей отрицательные значения на всех ненулевых векторах подпространства $\left\{(x, y, z) \in \mathbb{R}^{3} \mid x+y-2 z=0\right\}$. Ответ представьте в стандартном виде многочлена 2-й степени от координат $x, y, z$
\end{tcolorbox}
Положим, что $U=\{(x,y,z)\in\mathbb{R}^3\vert x+y-2z=0\}$\\[2mm]
\indent Возьмем базис $\e=(e_1,e_2,e_3)$, такой что
$$\begin{aligned}
    e_1=(1,-1,0)\\
    e_2=(2,0,1)\\
    e_3=(1, 0, 0)
\end{aligned}$$
\indent В этом базисе квадратичная форма имеет матрицу
$$B(Q,\e)=\begin{pmatrix}
    -1&0&0\\
    0&-1&0\\
    0&0&1
\end{pmatrix}$$
\indent Она будет неопределнной, так как в матрице присутствуют $-1\text{ и }1$\\[2mm]
\indent Ограничим $Q$ на данное подпространство $U$
$$Q\vert_U=\begin{pmatrix}
    -1&0\\
    0&-1
\end{pmatrix}$$
\indent Пусть $C$ — матрица перехода от стандартного базиса к базису $\e$, тогда $C$ имеет вид
$$C=\begin{pmatrix}
    1&2&1\\
    -1&0&0\\
    0&1&0
\end{pmatrix}$$
\indent Тогда, $B'$ — матрица формы $Q$ в стандартном базисе, причем $$B'=(C^{-1})^TBC^{-1},$$ из этой матрицы мы получим требуемый многочлен\\[2mm]
\begin{tcolorbox}[colback=yellow!20!white, colframe=black!100!black]
    \textbf{Пример.} $$\begin{aligned}
        Q(x,y,z)&=-x^2-y^2-z^2+(x+2-2z)^2\\
        Q(0,0,1)&=3\geqslant0
    \end{aligned}$$
\end{tcolorbox}
\newpage

\begin{tcolorbox}[colback=blue!20!white, colframe=black!100!black]
    \textbf{2022/2023. Вариант 1. Задача 3.} В евклидовом пространстве $\mathbb{R}^{3}$ со стандартным скалярным произведением даны векторы
$$u_{1}=(1,-1,2), u_{2}=(1,1,-1), u_{3}=(1,0,-1)$$
Обозначим через $v_{1}, v_{2}, v_{3}$ ортогональные проекции вектора $v=(5,3,-1)$ на подпространства $u_{1}^{\perp}, u_{2}^{\perp}, u_{3}^{\perp}$ соответственно. Найдите объём параллелепипеда, натянутого на векторы $v_{1}, v_{2}, v_{3}$
\end{tcolorbox}
$\pr_{u_i^{\perp}} v=v-\ort_{u_i^{\perp}} v=v-\pr_{\langle u_i\rangle} v=v-\displaystyle\frac{(v, u_i)}{(u_i,u_i)}u_i$\\[2mm]
\indent Применяя эту формулу получим три вектора\\[2mm]
\indent Далее считаем объем трехмерного параллелепипеда по формуле (например, как определитель матрицы $3\times3$)\footnote{Рома не дорешал :)}\\[2mm]

 
\newpage

\begin{tcolorbox}[colback=blue!20!white, colframe=black!100!black]
     \textbf{2022/2023. Вариант 1. Задача 4.} Приведите пример недиагонализуемого линейного оператора $\varphi$ в $\mathbb{R}^{2}$, для которого оператор $\varphi^{2}+3 \varphi$ диагонализуем.
\end{tcolorbox}
Возьмем жорданову клетку
$$A=\begin{pmatrix}
    \lambda&1\\
    0&\lambda
\end{pmatrix}\Rightarrow A^2+3A=\begin{pmatrix}
    \lambda^2&2\lambda\\
    0&\lambda^2
\end{pmatrix}+\begin{pmatrix}
    3\lambda&3\\
    0&3\lambda
\end{pmatrix}=\begin{pmatrix}
    \lambda^2+3\lambda&2\lambda+3\\
    0&\lambda^2+3\lambda
\end{pmatrix}$$
\indent При $2\lambda+3=0$ получаем диагонализуемый линейный оператор, то есть $\lambda=-\displaystyle\frac{3}{2}$

\newpage

\begin{tcolorbox}[colback=blue!20!white, colframe=black!100!black]
     \textbf{2022/2023. Вариант 1. Задача 5.} Вставьте вместо звёздочки, ромбика и кружочка подходящие числа таким образом, чтобы линейный оператор $\varphi: \mathbb{R}^{3} \rightarrow \mathbb{R}^{3}$, имеющий в стандартном базисе матрицу

$$
\left(\begin{array}{ccc}
\star & 2 / 3 & 2 / 3 \\
-2 / 3 & 1 / 3 & \diamond \\
2 / 3 & 2 / 3 & \circ
\end{array}\right)
$$

был ортогональным. Найдите ортонормированный базис, в котором матрица оператора $\varphi$ имеет канонический вид, и выпишите эту матрицу. Укажите ось и угол поворота, определяемого оператором $\varphi$.
\end{tcolorbox}
Столбцы (строки) этой матрицы должны образовывать ортонормированный базис\\[2mm]
\indent $2/3\cdot\star\cdot\displaystyle\frac{2}{9}+\frac{4}{9}=0=\Rightarrow \star=-\frac{1}{3}$\\[2mm]
\indent $\diamond=-\displaystyle\frac{2}{3}$\\[2mm]
\indent $\circ=-\displaystyle\frac{1}{3}$
\newpage

\begin{tcolorbox}[colback=blue!20!white, colframe=black!100!black]
     \textbf{2022/2023. Вариант 1. Задача 6.} Существует ли матрица $A \in \operatorname{Mat}_{2 \times 3}(\mathbb{R})$ ранга 2 со следующими свойствами:

1) одно из сингулярных значений матрицы $A$ равно $\sqrt{10}$;

2) ближайшая к $A$ по норме Фробениуса матрица ранга 1 есть $B=\left(\begin{array}{ccc}1 & -1 & 2 \\ -2 & 2 & -4\end{array}\right)$ ?

Если существует, то предъявите такую матрицу.
\end{tcolorbox}
Представим матрицу $B$ в виде произведения столбца на строку и нормируем
$$B=\begin{pmatrix}
    1\\
    -2
\end{pmatrix}\begin{pmatrix}
    1&-1&2
\end{pmatrix}=\sqrt{30}\begin{pmatrix}
    \frac{1}{\sqrt{5}}\\[1mm]
    -\frac{2}{\sqrt{5}}
\end{pmatrix}\begin{pmatrix}
    \frac{1}{\sqrt{6}}&-\frac{1}{\sqrt{6}}&\frac{2}{\sqrt{6}}
\end{pmatrix}=\begin{pmatrix}
    \frac{1}{\sqrt{5}}\\[1mm]
    -\frac{2}{\sqrt{5}}
\end{pmatrix}\sqrt{30}\begin{pmatrix}
    \frac{1}{\sqrt{6}}&-\frac{1}{\sqrt{6}}&\frac{2}{\sqrt{6}}
\end{pmatrix}$$
\indent Значит, существует требуемая матрица $A$, так как $\sqrt{10}<\sqrt{30}$\\[2mm]
$$\begin{aligned}
    A&=u_1\sigma_1 v_1^T&+u_2\sigma_2 v_2^T\\
    &=B&+u_2\sigma_2 v_2^T
    \end{aligned}$$
\indent Тогда пусть $\sigma_2=\sqrt{10}$\\[2mm]
\indent Выберем $u_2$ и $v_2$ так
$$\begin{aligned}
u_2&=\begin{pmatrix}
    \frac{2}{\sqrt{5}}&\frac{1}{\sqrt{5}}
\end{pmatrix}\\
v_2&=\frac{1}{\sqrt{2}}\begin{pmatrix}
    1&1&0
\end{pmatrix}\end{aligned}$$
\indent Остается только перемножить и сложить по формуле

\newpage

\begin{tcolorbox}[colback=blue!20!white, colframe=black!100!black]
     \textbf{2022/2023. Вариант 1. Задача 7.} Найдите прямоугольную декартову систему координат в $\mathbb{R}^{3}$ (выражение старых координат через новые), в которой уравнение поверхности

$$
3 x^{2}+2 y^{2}-4 x z-4 y+7=0
$$

имеет канонический вид. Укажите этот вид, определите тип поверхности и нарисуйте её эскиз.
\end{tcolorbox}
\indent Выделяем квадратичную форму
$$(x,y,z):\begin{pmatrix}
    3&0&-2\\
    0&2&0\\
    -2&0&0
\end{pmatrix}$$
\indent Ограничиваем ее
$$(x,z):\begin{pmatrix}
    3&-2\\
    -2&0
\end{pmatrix}$$
\indent А дальше ничего никто не сказал :)\\[2mm]
% См. консультацию Алены и мой \TeX\ ее консы \href{https://drive.google.com/file/d/1iV3cNWS9gDKkPBHholCHBuGCmXng0qlg/view?usp=sharing}{тут} и в чате

% $2y^2-4y=2(y-1)^2-2$









\end{document}
